\documentclass{article}

\usepackage{amsmath}

\title{Week 1 Report: Supervised Learning review}
\date{2017-04-27}
\author{Lam Dang}

\begin{document}
\pagenumbering{gobble}
\maketitle
\newpage
\pagenumbering{arabic}

\section{Supervised Learning}
Supevised Learning is a Statiscal Learning technique in which for each observation of the predictor mesurement $x_i$ there is an associated respond measurement $y_i$. The target is to fit a model that related the relation between the predictor and the respond for prediction or inference.
\section{Description of Linear Model}
\subsection{Model Form}
\begin{equation*}
Y=F(X)+\varepsilon=\beta_0 + \beta_1*X_1 + ... + \beta_n X_n + \varepsilon
\end{equation*}
where Y is the respond, X is the predictor. The Linear Model assume the linear relationship between the predictor X and respond Y
\subsection{Normal Equation}
In order to solve the weights for linear regression problem, we compute the least square line/plane. The least square vector is the solution of the normal equation:
\begin{equation*}
(X^TX)\hat{\beta} = X^TY
\end{equation*}
\subsection{Geometric Interpretation of the solution}

\subsection{Computation of the Prediction of a Linear Model}
After the weight for a linear system have been calcualted, prediction $\hat{Y}$ can be obtained by
\begin{equation*}
\hat{Y} = \hat{F}(X),
\end{equation*}
where $\hat{F}$ represent the estimation of $F(X)$. The form of $\hat{F}(X)$ is not relevant as long as it yield accurate $\hat{Y}$
\end{document}